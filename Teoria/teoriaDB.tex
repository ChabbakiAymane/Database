\documentclass[10pt]{article}
\usepackage[english]{babel}
\usepackage{amsmath,amsfonts,amssymb,mathtools,ulem,booktabs,subcaption,float}
\usepackage[utf8x]{inputenc}
\usepackage[margin = 0.75 in]{geometry}
\title{Database: Teoria}
\author{Aymane Chabbaki}
\date{III semestre 2019/2020}

\begin{document}
\maketitle
\tableofcontents
\newpage

\section{Relational Algebra}
	\subsection{}
	\begin{itemize}
	\item
	\end{itemize}	
%--------------------------------------------------------------------------------------------------%
\section{SQL}
	\subsection{}
	\begin{itemize}
	\item
	\end{itemize}
%--------------------------------------------------------------------------------------------------%
\section{Functional Dependencies}
	\subsection{}
	\begin{itemize}
	\item
	\end{itemize}
%--------------------------------------------------------------------------------------------------%
\section{Indexes}
	\subsection{}
	\begin{itemize}
	\item
	\end{itemize}
%--------------------------------------------------------------------------------------------------%
\section{Query Cost}
	\subsection{Page}
	\begin{itemize}
	\item
	Every time a Hard Drive is instructed to \textbf{read} or \textbf{write} something on the disk, it does so by \textbf{reading} or \textbf{writing units of specific size}.
	\item
	This \textbf{size} is called a \textbf{Page} and has always a \textbf{fixed size}.
	\item
	The \textbf{size of a page} on the disk will be denoted as $P$.
	\end{itemize}

\subsection{Size of a Record}
	\begin{itemize}
	\item
	\textbf{Records} of a relation have the \textbf{same size}.
	\item
	The \textbf{size of a record}:
		\begin{itemize}
		\item
		indicates how much space, in \textbf{bytes}, a \textbf{record occupies} when stored on the disk.
		\item
		will be typically \textbf{given} or it would be possible to \textbf{compute} from the \textbf{size of the individual attributes}.
		\end{itemize}
	\item
	The \textbf{size of a record} of a \textbf{relation R} will be denoted as $t_R$.
	\item
	Example:
		\begin{itemize}
		\item
		Student(ssn:\textbf{int}, credits:\textbf{int}, age:\textbf{int}, name:\textbf{varchar(25)}, surname:\textbf{varchar(25)}
		\item 
		Knowing that an \textbf{int} variable occupies $4$ bytes and a \textbf{varchar} variable occupies $1$ byte, we have that: $$t_{Student} = (3 \cdot 4) + (2 \cdot 25) = 62 \textrm{ bytes}$$
	\item
	So the size of each record of Student occupies $62$ bytes on the disk.
	\end{itemize}
\end{itemize}

\subsection{Pages of Relation}
	\begin{itemize}
	\item
	When a \textbf{page contains} some \textbf{data of a relation}, \textbf{no records from other relations are allowed} in that page.
	\item
	The database try to\textbf{ fill a page with as many records of the same relations as it can}, and if no more records can fit, then it start saving them in another page.
	\item
All the \textbf{pages} that a \textbf{relation occupies} on the disk are \textbf{full}.
	\item
The \textbf{number of pages} that a \textbf{relation R occupies} on the disk will be denoted as $P_R$.
	\end{itemize}

\subsection{Cardinality of a Relation}
	\begin{itemize}
	\item
	A relation is a set of records.
	\item
	The \textbf{cardinality} of a \textbf{relation R} will be denoted as $|R\,|$ (\textbf{number of records} a relation has).
	\end{itemize}

\subsection{Cardinality of an Attribute}
	\begin{itemize}
	\item
	The \textbf{cardinality} of:
		\begin{itemize}
		\item
		an \textbf{attribute} is the \textbf{number of different distinct values} that the attribute has.
		\item
		\textbf{two or more attributes} is the \textbf{number of different distinct combinations} of the values of these attributes.
		\end{itemize}	 
	\item
	The \textbf{cardinality} of:
		\begin{itemize}
		\item
		of an attribute $A$ of a relation R, will be denoted as $|R.A\,|$.
		\item
		of two or more attributes of a \textbf{relation $R$}, will be denoted as $| R.A_1, R.A_2, \dots, R.A_n\,|$.
		\end{itemize}
	\item
	If the attribute is a \textbf{key}, the \textbf{cardinality} of the attribute is the \textbf{same} as the cardinality of the \textbf{relation}:
		\begin{itemize}
		\item
		If the attribute $A$ is key, then $|R\,| = |R.A\,|$, otherwise $|R.A\,| \leq |R\,|$.
		\end{itemize}
	\end{itemize}

\subsection{Records per Page}
	\begin{itemize}
	\item
	\end{itemize}

\subsection{Relation size}
	\begin{itemize}
	\item
	\end{itemize}

\subsection{Cost}
	\begin{itemize}
	\item
	\end{itemize}

\subsection{Scan}
	\begin{itemize}
	\item
	\end{itemize}

\subsection{Sorting}
	\begin{itemize}
	\item
	\end{itemize}

\subsection{Indexes}
	\begin{itemize}
	\item
	\end{itemize}
%--------------------------------------------------------------------------------------------------%
\section{Transaction}
	\subsection{}
	\begin{itemize}
	\item
	\end{itemize}

\end{document} 
